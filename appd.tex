\chapter{Encoding conversion}\label{app:enc}

\begin{figure}[h]
    \caption{Code for generating encodings described in Table~\ref{tbl:features}\label{appfig:features-code}}
\begin{minted}{python}
"""
vec[0] = d1.x
vec[1] = d1.y
vec[2] = d2.x
vec[3] = d2.x
vec[4] = d3.x
vec[5] = d3.x
vec[6] = pen_down
vec[7] = pen_up
vec[8] = end
"""
def rotate(pt, theta):
    new_x = pt.x * math.cos(theta) - pt.y * math.sin(theta)
    new_y = pt.x * math.sin(theta) + pt.y * math.cos(theta)
    return Point(new_x, new_y)

def enc_A(bez, next_state):
    result = np.zeros(NUM_DIM)
    set_vec_state(result, next_state)
    set_d1(result, bez.end - bez.start)
    set_d2(result, bez.c1 - bez.start)
    set_d3(result, bez.c2 - bez.start)
    return result
\end{minted}
\end{figure}

\begin{figure}[h]
\begin{minted}{python}
def enc_B(bez, next_state):
    result = np.zeros(NUM_DIM)
    set_vec_state(result, next_state)
    set_d1(result, bez.c1 - bez.start)
    set_d2(result, bez.c2 - bez.c1)
    set_d3(result, bez.end - bez.c2)
    return result

def enc_C(bez, next_state):
    result = np.zeros(NUM_DIM)
    set_vec_state(result, next_state)
    disp = bez.end - bez.start
    theta = math.atan2(disp.y, disp.x)
    set_d1(result, disp)
    set_d2(result, rotate(bez.c1 - bez.start, theta))
    set_d3(result, rotate(bez.end - bez.c2, theta))
    return result

def enc_D(bez, next_state):
    result = np.zeros(NUM_DIM)
    set_vec_state(result, next_state)
    disp = bez.end - bez.start
    theta = math.atan2(disp.y, disp.x)
    set_d1(result, bez.end)
    set_d2(result, rotate(bez.c1 - bez.start, -theta))
    set_d3(result, rotate(bez.end - bez.c2, -theta))
    return result

def enc_E(bez, next_state):
    result = np.zeros(NUM_DIM)
    set_vec_state(result, next_state)
    set_d1(result, bez.end)
    set_d2(result, bez.c1)
    set_d3(result, bez.c2)
    return result
\end{minted}
\end{figure}

\clearpage
\newpage
